\documentclass[conference]{IEEEtran}
\IEEEoverridecommandlockouts
% The preceding line is only needed to identify funding in the first footnote. If that is unneeded, please comment it out.
\usepackage{cite}
\usepackage{amsmath,amssymb,amsfonts}
\usepackage{algorithmic}
\usepackage{graphicx}
\usepackage{textcomp}
\usepackage{xcolor}
\def\BibTeX{{\rm B\kern-.05em{\sc i\kern-.025em b}\kern-.08em
    T\kern-.1667em\lower.7ex\hbox{E}\kern-.125emX}}
\begin{document}

\title{Exploring market power using reinforcement learning for intelligent bidding strategies\\
\thanks{EPSRC}
}

\author{\IEEEauthorblockN{Alexander J. M. Kell, Matthew Forshaw, A. Stephen McGough}
\IEEEauthorblockA{\textit{School of Computing} \\
\textit{Newcastle University}\\
Newcastle, U.K. \\
\{a.kell2, matthew.forshaw, stephen.mcgough\}@newcastle.ac.uk}
%\and
%\IEEEauthorblockN{2\textsuperscript{nd} Given Name Surname}
%\IEEEauthorblockA{\textit{dept. name of organization (of Aff.)} \\
%\textit{name of organization (of Aff.)}\\
%City, Country \\
%email address or ORCID}
%\and
%\IEEEauthorblockN{3\textsuperscript{rd} Given Name Surname}
%\IEEEauthorblockA{\textit{dept. name of organization (of Aff.)} \\
%\textit{name of organization (of Aff.)}\\
%City, Country \\
%email address or ORCID}
%\and
%\IEEEauthorblockN{4\textsuperscript{th} Given Name Surname}
%\IEEEauthorblockA{\textit{dept. name of organization (of Aff.)} \\
%\textit{name of organization (of Aff.)}\\
%City, Country \\
%email address or ORCID}
%\and
%\IEEEauthorblockN{5\textsuperscript{th} Given Name Surname}
%\IEEEauthorblockA{\textit{dept. name of organization (of Aff.)} \\
%\textit{name of organization (of Aff.)}\\
%City, Country \\
%email address or ORCID}
%\and
%\IEEEauthorblockN{6\textsuperscript{th} Given Name Surname}
%\IEEEauthorblockA{\textit{dept. name of organization (of Aff.)} \\
%\textit{name of organization (of Aff.)}\\
%City, Country \\
%email address or ORCID}
}

\maketitle

\begin{abstract}

Abstract goes here.

% Background

% Methodology

% Results



\end{abstract}

\begin{IEEEkeywords}
reinforcement learning, bidding strategy, multi-agent system, electricity markets
\end{IEEEkeywords}

\section{Introduction}

- What is market power? \\
- Market power in decentralized electricity markets \\
- Increase in cost to the consumer \\


%- Why the current approaches / systems don't solve the problem


%- What is the innovation in this work

- Application of DDPG (continuous action space) \cite{Kell2020} \\
- Use of agent based model\\


%- A short description of the solution



%- What are the key take-home messages



% Contributions of this work


%- Outline of the rest of the work




\section{Literature Review}

- Applications of reinforcement learning to bidding strategies \\
- Reinforcement learning in energy markets \\


\section{Material}

- Market Structure of ElecSim (yearly outlook) \\
- Introduction to RL and DDPG (model-free approach and continuous action space)

\section{Methodology}

- Grouping agents based upon size and seeing results\\
- Observation and action space \\
- Allowing them to bid maximum of \textsterling600 and a market cap of \textsterling150



\section{Results}

- Show time-steps vs. reward for both scenarios \\
- Show the step change in reward after a certain amount of controlled capacity

\section{Discussion}

- Make suggestions based upon optimal level of competition \\
- Importance of understanding market power and having a regulator otherwise prices can significantly increase

\section{Conclusion}

- Future work (withold capacity)


\section{Acknowledgment}

This work was supported by the Engineering and Physical Sciences Research Council, Centre for Doctoral Training in Cloud Computing for Big Data [grant number EP/L015358/1].


%\section*{References}

\bibliographystyle{IEEEtran}
\bibliography{library,custom_bib}


\end{document}
